% ==============================================================================
%
%                             Conclusion
%
% ==============================================================================
\chapter{Conclusion}
In this last chapter the project results and problems are briefly summarized and
a short outlook for future work is given.

% ==============================================================================
%
%                             IP
%
% ==============================================================================
\section{Image Processing}
The implemented image processing algorithm performs an edge detecting Sobel
filter on an input image of varying size. It is capable of 100 MB/s
throughput at only one percent FPGA resource utilization. The IP block can be
duplicated multiple times on the FPGA to increase the throughput. Although the
throughput seems high, the Sobel operation does not show the real potential of
using FPGA for image processing. Using more FPGA resources a more complex
algorithm could be implemented yielding a similar performance in throughput.

% ==============================================================================
%
%                             COMM
%
% ==============================================================================
\section{Communication}
On the communication part of the project a file transfer protocol on top of UDP
is defined and implemented in hardware. File transfers of up to 1MB are tested
and a throughput of 8.93 MB/s was achieved. This corresponds to 98.9\% of the
theoretical maximum. It is important to note that this was only achieved by
delaying the packets by 100$\mu s$ to avoid packet loss. This was necessary
because packet retransmission upon no acknowledgment is not yet implemented.

% ==============================================================================
%
%                             Together
%
% ==============================================================================
\section{Future Work}
For further steps a more demanding image processing task could be implemented to
show the advantages of using FPGA for image processing. An example would be the
\textit{Canny Edge Detector} that takes the output of Sobel and creates a binary
image with better edge detection. Furthermore the communication can be improved
by implementing the packet acknowledgment for reliable and faster data transfer.
Finally the workload could be distributed onto multiple FPGAs on a network
 to further increase throughput.
