% ==============================================================================
%
%                             Conclusion
%
% ==============================================================================
\chapter{Conclusion}
In this last chapter the project results are briefly summarized and
a short outlook for possible future work is given.

% ==============================================================================
%
%                             IP
%
% ==============================================================================
\section{Image Processing}
A new image processing core was implemented that locally optimizes contrast 
(Wallis filter). Multiple implementations have been validated and compared.
Using high level synthesis to describe an algorithm has proven that the desired
operation can be achieved in little time. Nevertheless the theoretical possible
throughput could not be achieved due to the dataflow not being
described well enough in C/C++ language to achieve the desired throughput.
Therefore a VHDL implementation was written that processes pixels at 125Mp/s and
uses approximately three percent of FPGA ressources.

% ==============================================================================
%
%                             Dataflow
%
% ==============================================================================
\section{Dataflow}
The existing communication core was extended with acknowledge, user registers
and AXI4-Stream interfaces. The new image processing algorithm requires the
input data in a specific order. Therefore a controller was implemented using two
different design flows. The HLS approach showed that complex interfaces such as
AXI4 can be implemented with a few lines of code and that state machines can be
implemented as well. The unfavourable memory management and the lack of high
throughput
led to the implementation of a VHDL based controller and memory management unit.
It caches necessary image data to obviate multiple image transmissions and can
support the full throughput of the image processing core.

% ==============================================================================
%
%                             Overall
%
% ==============================================================================
\section{Overall}
The final product called \gls{diip} is based on the VHDL implementations of
image processing and dataflow parts. Images can be sent from the PC to the FPGA
where
the Wallis operation is applied and the image is sent back. Image data
throughput of up to 4.1MB/s have been measured. The system is designed with
scalability in mind. A dedicated theoretical examination shows that if the
image processing core was implemented 21 times in the FPGA the full bandwidth of
gigabit Ethernet could be used to process image data yielding 125MB/s
throughput. Furthermore if multiple FPGAs were used, the total throughput would
scale proportional to the number of FPGAs used.

% ==============================================================================
%
%                             Working with High Level Synthesis
%
% ==============================================================================
\section{Working with High Level Synthesis}
The time spent on working with \gls{vivadohls} has shown several advantages and
disadvantages. It has shown that thinking close to hardware is crucial. Using
C/C++ as language is a trap to fall back and think of the code as if it would be
executed sequentially. The function to be implemented should be split into
building blocks the same way as it would be done using a hardware description
language. If the code was written with the exact same line of thought as a HDL
approach, then the same throughput should be achievable. And if that is managed,
HLS would bring significant advantages like the simple implementation of complex
interfaces (e.g. AXI4) and the reduced time in testbench development.

% ==============================================================================
%
%                             Future Work
%
% ==============================================================================
\section{Future Work}
The scalability is proven on paper and can now be implemented onto the FPGA.
The controller core can be extended to send the data to multiple Wallis filter
cores. After writing a computer application that can handle the fast transfer
speeds the true benefit of using FPGA for image processing can be shown in
praxis. Because the data transfer is Ethernet based a cluster of FPGAs can be
built to compete against high performance CPU and GPU based image processing
pipelines. Another method to increase throughput would be to implement a VHDL
256bit image processing core that would be able to process 21 pixels per clock
cycle and produce one pixel per clock cycle on the output. This would omit the
scalability inside the FPGA because the Ethernet link would already be
saturated. Furthermore the HLS implementations could be improved by describing
the dataflow more hardware oriented and prove that using HLS the same
throughput can be achieved as when using a hardware description language.

