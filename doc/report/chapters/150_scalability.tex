% ==============================================================================
%
%                             Scalability
%
% ==============================================================================
\chapter{Scalability} \label{chapt:scalability}
The code for this project is written from the beginning with the possibility to
scale it up in mind. The main idea being that the throughput can be increased.
Scalability in regard to this project can be divided into two terms:
\begin{itemize}
    \item Inside FPGA
    \item Across multiple FPGAs
\end{itemize}

In the following sections these two aspects are dissected theoretically.

\section{Inside FPGA} \label{chapt:sca:inside}
Every FPGA has a given amount of ressources (LUTs, memory and so forth, as
described in table \ref{tab:XC7A200T}). The inside FPGA scalability aims towards
optimal usage of these ressources. Before methods for scalability can be
compared it must be clarified what can be scaled. The \gls{diip} project
consists of three main parts. The communication, controller and image processing
parts. The communication part is only implemented once to handle the
communication to the PC and therefore can not be scaled. The image processing
part however can be implemented multiple times inside the FPGA to increase
throughput. Therefore the conrtoller part splits up the incomming image data and
distributes it to the image processing cores. To summarize the inside FPGA
scalability:
\begin{itemize}
    \item Implement the communication part once
    \item Implement the image processing part as many times as the available
    ressources allow it
    \item Adapt the controller to distribute data accross the processing cores
\end{itemize}

Figure \ref{fig:insidefpgascaleconceptbd} clarifies the concept. Note that all
image processing cores are equal.

\begin{figure}[tb!]
    \centering
    \begin{adjustbox}{max width=\linewidth}
        % \tikzsetnextfilename{system-overview}
\begin{tikzpicture}[
    rounded corners=0mm,
]
    %coordinates
    \coordinate (corig)      at (0,0);
    \coordinate (cmonitor)   at (0,0);
    \coordinate (ccom)       at (5,0);
    \coordinate (cip)        at (10,0);


    %nodes

    \begin{pgfonlayer}{main}

        \node[draw, fill=white, minimum width=3cm, minimum height=2cm, anchor=west, text width=2.8cm, align=center] (com) at (ccom) {Controller and Image Cache};

        \node[draw, fill=white, minimum width=3cm, minimum height=1cm, anchor=west, text width=2.8cm, align=center, above =1cm of com] (commu) {Communication};

        \node[draw, fill=white, minimum width=3cm, minimum height=1cm, anchor=west, text width=2.8cm, align=center, right = 1cm of com, yshift=2.5cm] (ip1) {Image\\Processing};
        \node[draw, fill=white, minimum width=3cm, minimum height=1cm, anchor=west, text width=2.8cm, align=center, right = 1cm of com, yshift=1.0cm] (ip2) {Image\\Processing};
        \node[draw, fill=white, minimum width=3cm, minimum height=1cm, anchor=west, text width=2.8cm, align=center, right = 1cm of com, yshift=-0.5cm] (ip3) {Image\\Processing};
        
        \node[circle,fill=black,minimum size=0.2cm,inner sep=0pt, below = 0.3cm of ip3] (dt1)  {};
        \node[circle,fill=black,minimum size=0.2cm,inner sep=0pt, below = 0.2cm of dt1] (dt2)  {};
        \node[circle,fill=black,minimum size=0.2cm,inner sep=0pt, below = 0.2cm of dt2] (dt3)  {};

        % \node[] (eth) at ($(cmonitor) + (4.5, 1.0)$) {LAN};
        
        % \draw[line width = 0.5mm] ($(eth) + (0,-1.0)$) ellipse (0.2cm and 0.5cm);
    \end{pgfonlayer}

    % FPGA box
    \begin{pgfonlayer}{main}
        \node[above = 2.4cm of com, xshift=-1.2cm] (fpga) { FPGA };
    \end{pgfonlayer}
    \begin{pgfonlayer}{foreground}
        \node (f_fpga) [draw=black, fill=gray!20, inner sep=10, fit={(com) (ip1) (ip2) (ip3) (dt2) (dt1) (dt3)}] {};
    \end{pgfonlayer} 

    
    \path[draw,{Latex[length=4mm]}-{Latex[length=4mm]},line width =1mm] ($(commu.180) + (-1.5,0)$) -- ($(commu.180) + (0,0)$) 
        node[near start, left, anchor=east,xshift=-0.5cm] () {PC} ;
    \path[draw,{Latex[length=4mm]}-{Latex[length=4mm]},line width =1mm] ($(com.90) + (0,0)$) -- ($(commu.270) + (0,0)$) node[near start, left, anchor=east,xshift=-0.5cm] () {} ;

    \path[draw,{Latex[length=2.5mm]}-{Latex[length=2.5mm]}] 
        ($(com.0) + (0,0.7)$) -| ($(ip1.180) + (-0.6,0)$) -- ($(ip1.180) + (0,0)$)
         node[near start, left, anchor=east,xshift=-0.5cm] () {} ;
    \path[draw,{Latex[length=2.5mm]}-{Latex[length=2.5mm]}] 
        ($(com.0) + (0,0.1)$) -| ($(ip2.180) + (-0.4,0)$) -- ($(ip2.180) + (0,0)$) 
        node[near start, left, anchor=east,xshift=-0.5cm] () {} ;
    \path[draw,{Latex[length=2.5mm]}-{Latex[length=2.5mm]}] ($(com.0) + (0,-0.5)$) -- ($(ip3.180) + (0,0)$) node[near start, left, anchor=east,xshift=-0.5cm] () {} ;

    % Legend
    \path[draw,{Latex[length=4mm]}-{Latex[length=4mm]},line width =1mm] 
        ($(ip1.0) + (1,0)$) -- ($(ip1.0) + (2,0)$) 
        node[near end, right, anchor=west,xshift=0.5cm] () {High bandwidth} ;
    \path[draw,{Latex[length=2.5mm]}-{Latex[length=2.5mm]}] 
        ($(ip1.0) + (1,-0.50)$) -- ($(ip1.0) + (2,-0.50)$) 
        node[near end, right, anchor=west,xshift=0.5cm] () {Low bandwidth} ;

\end{tikzpicture}
    \end{adjustbox}
    \caption{Block diagram of a inside FPGA scaled solution}
    \label{fig:insidefpgascaleconceptbd}
\end{figure}

Now that the concept is clarified, two possible solutions are compared to
distribute the image data accross the image processing cores, proposal A and B.
They differ in the way in what order the data is sent to the FPGA and how it is
cached inside the FPGA. For each proposal the following metrics are calculated.

\begin{description}
    \item[Initial size $s_i$]\hfill \\
    The number of pixels that have to be sent to the FPGA before beginning
    interational operation.
    \item[Iteration size $s_r$]\hfill \\
    How many pixels that have to be sent to start a new iteration.
    \item[Store size $s_s$]\hfill \\
    How many pixels have to be cached inside the FPGA.
    \item[Number of inits per image $n_i$]\hfill \\
    Denotes the number of initial data transfers of size $s_i$ have to be made
    per image.
    \item[Total tx size $s_{tx}$]\hfill \\
    The total number of pixels sent to the FPGA to calculate one image.
\end{description}


\begin{table}[tb!]
    \centering
    \begin{tabular}{p{0.45\textwidth} p{0.45\textwidth}}
        \toprule
        \multicolumn{1}{c}{Solution A} & \multicolumn{1}{c}{Solution B} \\
        \midrule
        Insert figure & Insert figure \\\midrule
        \textbf{Concept} & \\
        Image data is sent line wise. Each image processing core handles one
        input line, starts as soon as enough data is transfered and jumps $N$ lines
        downwards after completion.
        &
        Image data is sent coloumn wise with $w_l+(N-1)$ pixels per coloumn. All
        image processing cores start processing at the same time and progress
        $N$ lines downwards after completion.
        \\\midrule
        \textbf{Initial size} & \\
        {\( 
            s_i = i_w(w_l+(N-1))
        \)}
        &
        {\( 
            s_i = w_l(w_l+(N-1))
        \)}
        \\\midrule
        \textbf{Iterarion size} & \\
        {\( 
            s_r  = i_w
        \)}
        &
        {\( 
            s_r  = w_l+(N-1)
        \)}
        \\\midrule
        \textbf{Store size} & \\
        {\( 
            s_s  = i_w(w_l+(N-1))
        \)}
        &
        {\( 
            s_s  = w_l(w_l+(N-1))
        \)}
        \\\midrule
        \textbf{Number of inits per image} & \\
        {\( 
            n_i  = 1
        \)}
        &
        {\( 
            n_i  = \frac{1}{N}(i_w-w_l+1)
        \)}
        \\\midrule
        \textbf{Total tx size} & \\
        {\( 
            s_{tx}  = i_w \cdot i_h
        \)}
        &
        {\( 
            s_{tx}  = n_i s_i + (i_w-w_l)s_r n_i
        \)}
        \\
        \bottomrule
    \end{tabular}
    \caption{Inside FPGA scalability methods}
    \label{tab:insidefpgascalability}
\end{table}

Before comparing the two proposals another metric has to be taken into
consideration: The effective throughput of the image processing core. The way
the wallis filter works is that it requires a neighbourhood of pixels. For each
line the core processes, the initial neighbourhood has to be resent. Equation
\ref{eq:theomaxvhdlwallise} derived in appendix 
\ref{app:derivations:theomaxvhdlwallis} is used to calculate the effective
throughput $b_r$ of a Wallis filter core. To simplify, the image height $i_h$ is
considered to be much larger than the window length $w_l$ and therefore the
equation can be noted as shown in equatio \ref{eq:vhdlwallismaxb}. Using the
throughput of the VHDL Wallis filter implementation ($b_w=125Mp/s$) and the
window length $w_l=21$ the effective bandwidth $b_r$ is calculated:

\begin{align}
    b_r  & \approx \frac{b_w}{w_l} = 5.95 Mp/s
    \label{eq:vhdlwallismaxb}
\end{align}

Under the assumption that the real throughput $b_r$ scales proportionally with the
number of image processing cores implemented, equation 
\ref{eq:scaledrealttotalhroughput} can be noted to represent the total image
processing throughput $b_t$.

\begin{align}
    b_t  & = N \cdot b_r
    \label{eq:scaledrealttotalhroughput}
\end{align}

Considering the maximum throughput of the Gigabit-Ethernet connection of
$b_e=125MB/s$ \footnote{True Ethernet throughput is less than 125MB/s
considering packet overhead} equation \ref{eq:maxethernetthrouhgput} can be derived to calculate the
maximum number of image processing cores that can be implemented before the
Ethernet communication link is saturated.

\begin{align}
    b_e \geq b_t = N \cdot b_r \Rightarrow N \leq \frac{b_e}{b_r} \approx 
    \frac{b_e w_l}{b_w} = w_l = 21
    \label{eq:maxethernetthrouhgput}
\end{align}

To calculate the free memory the results from \ref{asdf} \todo{ref to ressource
VHDL} are dissected to calculate the available memory. Table \ref{tab:membudget}
lists the required block memory usage.

\begin{table}[h!]
    \centering
    \begin{tabular}{l r}
        \toprule
        Item & BRAM tiles \\
        \midrule
        Total available & 365 \\
        TEMAC support & -2 \\
        UDP IP Core & -1.5 \\
        21 divider generators & -10.5 \\
        21 Wallis cores & -10.5 \\
        \midrule
        Free & 340.5\\
        \bottomrule
    \end{tabular}
    \caption{FPGA block memory budget}
    \label{tab:membudget}
\end{table}

After using up memory for the communication and image processing cores, 340.5
block RAM tiles with 36Kb each results in 1'532KB of remaining block RAM that
can be used for image caching. Table \ref{tab:parsum} summarizes the
parameters that are used to compare the two inside FPGA scale proposals.

\begin{table}[h!]
    \centering
    \begin{tabular}{l r l}
        \toprule
        Parameter & Value & Description\\
        \midrule
        $N$ & 21 & Number of image processing cores to implement \\
        $s_b$ & 1'532KB & Block memory storage available for cache \\
        $w_l$ & 21 & Window length of the Wallis filter \\
        \bottomrule
    \end{tabular}
    \caption{Parameter summary}
    \label{tab:parsum}
\end{table}


\section{Across FPGA}
text

\section{Usecase}

