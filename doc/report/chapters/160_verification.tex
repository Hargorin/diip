% ==============================================================================
%
%                             Verification & Benchmark
%
% ==============================================================================
\chapter{Verification and Benchmark} \label{chapt:ver_bench}
With the image processing and dataflow parts implemented, they can be verified
and benchmarked. The next chapters hold the verification process of both the
image processing part and the dataflow part. After both components are verified
they are benchmarked as one unity against a computer based implementation in
chapter \ref{ch:benchmark}. Before starting, the following chapter summarizes
all solutions developed in this project.
\clearpage

% ==============================================================================
%
%                             Overview
%
% ==============================================================================
\section{Overview} \label{ch:overview}
This chapter lists all the solutions that are compared in verification and benchmark. The solutions can be found in table \ref{tab:ver:overview}.

\begin{table}[h!]
    \centering
    \begin{tabular}{p{3.8cm} p{10cm}}
        % \toprule
        % Solution & Description \\
        \toprule
        \multicolumn{2}{l}{\textbf{Image Processing}}\\
        \toprule
        HLS 8bit & This solution is a Wallis filter with an 8bit AXI4-Stream as input and using C/C++ in Vivado HLS and is implemented on the FPGA.\\
        \midrule
        HLS 256bit & This solution is a Wallis filter with a 256bit AXI4-Stream as input and using C/C++ in Vivado HLS and is not implemented on the FPGA.\\
        \midrule
        VHDL 8bit & This solution is a Wallis filter with an 8bit AXI4-Stream as input and using VHDL and is implemented on the FPGA.\\
        \midrule
        C/C++ Wallis filter & This solution is a Wallis filter using C/C++ on the Computer.\\
        \bottomrule
        \multicolumn{2}{l}{\vspace{1ex}}\\
        \toprule
        \multicolumn{2}{l}{\textbf{Controller}}\\
        \toprule
        Controller HLS &  The HLS controller reads image data stored in block RAM using AXI4 interface and sends it to the Wallis filter via AXI4-Stream. Works only with memory based communication. \\
        \midrule
        Controller VHDL & The VHDL controller receives image data directly from the communication part via AXI4-Stream and sends it to the Wallis filter via AXI4-Stream. Works only with stream based communication. \\
        \bottomrule
        \multicolumn{2}{l}{\vspace{1ex}}\\
        \toprule
        \multicolumn{2}{l}{\textbf{Communication}}\\
        \toprule
        Communication Memory based & Stores the received image data in block RAM using AX4 interface. \\
        \midrule
        Communication Stream based & Received data is streamed directly to the controller. Data to send is also transferred using AXI4-Stream. \\
        \bottomrule
        \multicolumn{2}{l}{\vspace{1ex}}\\
        \toprule
        \multicolumn{2}{l}{\textbf{Overall}}\\
        \toprule
        diip & This solution includes the HLS 8bit Wallis filter, the HLS controller and the communication based on memory. \\
        \midrule
        diip\_fatser & This solution includes the VHDL 8bit Wallis filter, the VHDL controller and the communication based on streaming. \\
        \bottomrule
    \end{tabular}
    \caption{Overview of the solutions for the verification and benchmark}
    \label{tab:ver:overview}
\end{table}
\clearpage

% ==============================================================================
%
%                             Verification
%
% ==============================================================================
\section{Verification} \label{ch:verification}
The verification process ensures that all components work as expected. It is
split into the image processing and dataflow parts. They are tested
independantly to reduce complexity and simulation time. The system as a unity is
then tested in chapter \ref{ch:verification:overallvalidation}.

% ==============================================================================
%
%                             Image Processing
%
% ==============================================================================
\subsection{Image Procession}\label{ch:verification:imageprocessing}
The verification of the image processing is done using two different 
methods. First, an image is processed with the Wallis filter and compared 
with a reference image. The reference image is generated from a C/C++ program 
which calculates using floating point precision for calculations. In the second
aspect the 
throughput is validated. \\
The devices under validation (DUV) consist of two written in Vivado HLS using
C/C++
and one VHDL version. The Vivado HLS versions differ in the input stream. One 
has an 8 bit AXI4-Stream and the other one has a 256 bit AXI4-Stream input. The
VHDL has an 8 bit AXI4-Stream.

\subsubsection*{Image Comparsion}
During the verification an image is processed with the Wallis filter. The Wallis
filter is run on the computer as a C/C++ program and has floating point
variables. This Wallis filtering is used as a reference. Two images (room and
mountain), added in appendix \ref{app:images_wallis}, are chosen as source images. \\
The source images are passed through the \gls{duv} and compared with the reference
images. The \gls{rmse} serves as a metric. The RMSE
indicates how much the  Wallis filtered image deviates on average from the
reference image \cite{rmse}. The parameters for the Wallis filter that were used
for the images can be found in table \ref{tab:parameter}.
\\

\begin{table}[b!]
    \centering
    \begin{tabular}{l l l l l}
        \toprule
        Image & Brightness & Contrast & Global Mean & Global Variance \\
        \midrule
        room \ref{fig:ref_room} & 0.5 & 0.8125 & 127 & 3600 \\
        mountain \ref{fig:ref_mountain} & 0.5 & 0.8125 & 127 & 3600 \\
        \bottomrule
    \end{tabular}
    \caption{Parameters for the Wallis filter}
    \label{tab:parameter}
\end{table}

Table \ref{tab:rmse_room} lists the RMSE values of the room and mountain image. They
represent the deviation from the reference image in percent. The deviations from
the C/C++ program with the 8 bit AXI4 stream and the 256 bit AXI4 stream do
not differ from each other. A total deviation of 0.32\% has been measured for
the
room image. This corresponds to an intensity value of 0.816 in the range of 0 -
255. There is also only a deviation of 0.33\% in the mountain image. The VHDL
verification is with 1.23\% respectively 0.49\% higher than the
HLS versions. The intensity deviates by 3.14 or 1.25 pixel values respectively.
It is assumed that the deviation occurs due to rounding in the VHDL solution.

\pagebreak
In comparison to the HLS solution, the VHDL solution does not calculate the mean
and the variance exactly. Further it is assumed that image contents play a
role, meaning how strong the intensity in the image differs in regard to the
location. The room image changes intensity more often, whereby the mountain
image is bright at the top and dark at the bottom.

\begin{table}[tb!]
    \centering
    \begin{tabular}{l l l l l l l}
        \toprule
         & \multicolumn{2}{c}{C/C++ Simulation (HLS)} & \multicolumn{2}{c}{RTL Simulation (HLS)} & \multicolumn{2}{c}{VHDL Simulation (ghdl)} \\
        {image} & \multicolumn{1}{l}{room} & \multicolumn{1}{l}{mountain} & \multicolumn{1}{l}{room} & 
        \multicolumn{1}{l}{mountain} & \multicolumn{1}{l}{room} & \multicolumn{1}{l}{mountain} \\
        \midrule
        HLS  8bit    & 0.32\% & 0.33\%    & 0.32\% & 0.33\%  & {} & {}\\
        HLS  256bit  & 0.32\% & 0.33\%    & 0.32\% & 0.33\%  & {}& {} \\
        VHDL 8bit    & {} & {}            & {} & {}          & 1.23\%  & 0.49\%\\
        \bottomrule
    \end{tabular}
    \caption{RMSE of the Wallis filter verification}
    \label{tab:rmse_room}
\end{table}

% \begin{table}[tb!]
%     \centering
%     \begin{tabular}{l l l l}
%         \toprule
%         DUV & C/C++ Simulation (HLS) & RTL Simulation (HLS) & VHDL Simulation (ghdl) \\
%         \midrule
%         HLS  8bit    & 0.33\%    & 0.33\%    & {} \\
%         HLS  256bit  & 0.33\%    & 0.33\%    & {} \\
%         VHDL 8bit        & {}         & {}         & 0.49\% \\
%         \bottomrule
%     \end{tabular}
%     \caption{RMSE of the Wallis filter verification using the mountain image}
%     \label{tab:rmse_mountain}
% \end{table}

\subsubsection*{Throughput}
During verification, the throughput of the Wallis filter is also evaluated.
For this purpose, the throughput at the input of the filter is measured. This
means how many pixels per clock are loaded into the Wallis filter. Again, the
three different methods (HLS 8bit, HLS 256bit and VHDL 8bit) of the Wallis filter
are compared. Table \ref{tab:throughput} lists the three different methods. The
8bit versions should have a throughput of one pixel per clock, this is 125Mp/s
at 125MHz clock frequency. But the HLS 8 bit version only reaches 21 pixels in
86 clocks. This results in a throughput of 30.5Mp/s. It is four times slower
than the theoretical value. This has to do with the fact that the division
requires 23 clock cycles and that the loop cannot be pipelined by reading the
pixels (see chapter \ref{ch:hls:div} and \ref{ch:ip:axi}). For this reason the
256 bit version was developed. It has a theoretical maximum of 2625Mp/s
throughput. A throughput of 21 pixels per 11 clock cycles could be achieved.
This
is equivalent to 238.6Mp/s. In relation to the 8 bit version only a better
throughput of a factor of 7.8 could be achieved. The limiting factor is still
the dataflow in the image processing core of the pixel reading. With the VHDL version the theoretical maximum of 125Mp/s has been
reached.

\begin{table}[h!]
    \centering
    \begin{tabular}{l l l l}
        \toprule
        DUV & Target [p/clk] & Actual [p/clk]  & Throughput [Mp/s]\\
        \midrule
        HLS  8bit    & 1     & 21/86     & 30.5\\
        HLS  256bit  & 21    & 21/11     & 238.6 \\
        VHDL 8bit        & 1     & 1         & 125 \\
        \bottomrule
    \end{tabular}
    \caption{Throughput of the Wallis filter related to the input pixels at
    125MHz clock frequency}
    \label{tab:throughput}
\end{table}

\clearpage
\subsubsection*{Resource} \label{ch:ver:ip:ressource}
For the implementation of the Wallis filter, another important factor is the
resources it requires on the FPGA. The resources used on the FPGA for the
implementation of the Wallis filter are listed in table \ref{tab:ressource} as
absolute values and as percentage of the available on the FPGA (value in
parentheses).
\glspl{slice}, Block RAMs and DSPs are compared with each other. A slice on the FPGA 
(XC7A200T) contains four LUTs and eight flip-flops \cite{ds180}.
The VHDL implementation needs the least resources in the comparison. All
three implementations need practically the same number of DSPs. Comparing
the two HLS versions, the 256bit version needs less block memory, but more
slices.

\begin{table}[tb!]
    \centering
    \begin{tabular}{l l l l l l l}
        \toprule
        DUV         & Slice & (33650) & BRAM & (365) & DSP & (740) \\
        \midrule
        HLS  8bit    & 949 & (2.8\%)   & 10.5 & (2.9\%)  & 17 & (2.3\%) \\
        HLS  256bit  & 1542 & (4.6\%)  & 5 & (1.4\%)     & 16 & (2.2\%) \\
        VHDL 8bit       & 253 & (0.8\%)   & 1 & (0.3\%)     & 16 & (2.2\%) \\
        \bottomrule
    \end{tabular}
    \caption{Resources of the three implemented Wallis filters}
    \label{tab:ressource}
\end{table}


% ==============================================================================
%
%                             Dataflow
%
% ==============================================================================
\subsection{Dataflow}\label{ch:verification:dataflow}
The dataflow part is again divided into two parts, the communication and
controller parts. While in the communication part only the most recent version
is tested (the stream based communication, referring to chapter 
\ref{chapt:dataflow} explaining the solution
A and B with streaming interface), both versions of the controller are verified.

\subsubsection*{Communication}
The communication part has in a large part been taken from the last semester project
and has been thoroughly tested and validated in the project report 
\cite{p5report}. The three new implemented features are verified in this
chapter. They consist of:
\begin{itemize}
    \item Acknowledge
    \item User registers
    \item Stream interface
\end{itemize}

\vspace{1ex}
\textbf{Acknowledge:} To test the acknowledge function, a file was sent to the
FPGA from the computer
and the Ethernet traffic was monitored with Wireshark Network Protocol Analyzer.
Wiresharks packet dissections are exported to a json file that was then analyzed
using the \texttt{uftcheck} utility. It was written to analyze network packets
for UFT transfers. The first lines of output are shown in listing \ref{lst:uftcheckoutput}.

    % \begin{adjustbox}{max width=1.7\textwidth}
\begin{minipage}{1\linewidth}
    \begin{lstlisting}[
        style=TextStyle, 
        caption=Output of \texttt{uftcheck} for package acknowledgment, 
        label=lst:uftcheckoutput
        ]
+------+----------------------+----------------------+-----+------------+------+------+
| Pack |         From         |          To          | D/C |  Control   | TCID | SEQ  | 
+------+----------------------+----------------------+-----+------------+------+------+
|  1   | 192.168.5.10 (50719) | 192.168.5.9 (42042)  |  C  |  FT Start  |  12  | 1036 |
|  2   | 192.168.5.10 (50719) | 192.168.5.9 (42042)  |  D  |            |  12  |  0   |
|  3   | 192.168.5.9 (42042)  | 192.168.5.10 (50719) |  C  | ACK packet |  12  |  0   |
|  5   | 192.168.5.10 (50719) | 192.168.5.9 (42042)  |  D  |            |  12  |  1   |
|  6   | 192.168.5.9 (42042)  | 192.168.5.10 (50719) |  C  | ACK packet |  12  |  1   |
\end{lstlisting}
\end{minipage}
% \end{adjustbox}

\vspace{1ex}
The sending PC starts a file transmission with a file transfer start packet and
the first data packet. The next packet (3) is coming from the FPGA to the PC
and acknowledges the first data packet (sequence 0). To verify that all packets
are acknowledged, the sending program reports an acknowledge status after the
file has been sent. Listing \ref{lst:uftsenderout} shows the output of the UFT
sender program.

\begin{minipage}{1\linewidth}
    \begin{lstlisting}[
        style=TextStyle, 
        caption=UFT send console output, 
        label=lst:uftsenderout
        ]
$ ./sender 192.168.5.9 42042 payload/cat.jpg
UFT Sender demo
destination 192.168.5.9:42042
HURRAY! All 1036 packets have been acknowledged.
time elapsed: 1.18s Speed: 0.859 MB/s Size: 1.012 MB\end{lstlisting}
\end{minipage}

\vspace{1ex}
\textbf{User register:} To test whether the user registers can be written and are
output correctly by the communication core, an \gls{ila} 
was used. An ILA can be configured on the FPGA to record internal signals. The
results are transferred to the PC over USB and displayed in \gls{vivadohlx}. All
user registers (0 through 7) are written with different values and the result
was observed using the ILA.

\vspace{1ex}
\textbf{Stream interface:} The last modification made to the communication core
was the AXI4-Stream interface. This was similarly tested as the user registers.
Using an ILA the output of the stream was observed and the correct order of data
verified. 

\subsubsection*{Controller} 
The most important thing to verify in the controller core is the correct order
of output pixels. This is done by generating input data representing an image,
feeding this data through the controller core and observing the output pixels.
This validation is split into the two solutions implemented using HLS (solution
A) and VHDL (solution B).

\clearpage
\textbf{Solution A) HLS:} In Vivado HLS the validation was done in C/C++ and
RTL-simulation. A testbench C/C++ file generates pseudo random image data. This
data is then processed using the controller core and in the testbench itself.
The two results are then compared. Using different input data and image sizes,
all output results matched. A test on the FPGA was omitted. This validation will
later take place in the overall validation in chapter \ref{ch:verification:overallvalidation}.

\vspace{1ex}
\textbf{Solution B) VHDL:} The same approach was used in the solution written in
VHDL except that the testbench is not written in C/C++ but in VHDL. The input
stream sends an incrementing pixel value with an image width and height of
eight. The window length was reduced to three for simplicity. Figure 
\ref{fig:vhdlcontrollerstimuli} shows the input data values.

 \begin{figure}[t!]
    \centering
    \begin{adjustbox}{max width=\linewidth}
        % \tikzsetnextfilename{system-overview}
\begin{tikzpicture}[
    rounded corners=0mm,
    triangle/.style = {fill=blue!20, regular polygon, regular polygon sides=3 },
    node rotated/.style = {rotate=180},
    border rotated/.style = {shape border rotate=180}
]
    %coordinates
    \coordinate (orig)      at (0,0);

    \begin{pgfonlayer}{main}
        
        % Text
        % \node[] (write) at (-2,5) {Write};
        % \node[] (read) at (0,6.2) {Read};

        % Braces
        \draw [line width=0.5mm,decorate,decoration={brace,amplitude=10pt},xshift=-4pt,yshift=0pt] (4.5,4) -- (4.5,0) node [black,midway,xshift=0.5cm,anchor=west] {Height};
        \draw [line width=0.5mm,decorate,decoration={brace,amplitude=10pt},xshift=-0pt,yshift=4pt] (4,-0.5) -- (0,-0.5) node [black,midway,yshift=-0.5cm,anchor=north] {Width};
        
        % Center pixel
        \draw[black,line width=0.5mm] (0.5,3.0) rectangle (1,3.5);
        % Window size
        \draw[black,line width=0.5mm] (0,2.5) rectangle (1.5,4.0);

        % Axis
        \foreach \x in {0,1,2,3,4,5,6,7}
            \node[anchor=south] at ($(0.25,0) + (0,3.5)+0.5*(\x,0)$)  {0\padzeroes[1]\Hexadecimalnum{\x}};
        \foreach \x in {8,9,10,11,12,13,14,15}
            \node[anchor=south] at ($(0.25,0) + (-4,3.0)+0.5*(\x,0)$)  {0\padzeroes[1]\Hexadecimalnum{\x}};
        \foreach \x in {16,17,18,19,20,21,22,23}
            \node[anchor=south] at ($(0.25,0) + (-8,2.5)+0.5*(\x,0)$)  {\xintDecToHex{\x}};
        \foreach \x in {24,25,26,27,28,29,30,31}
            \node[anchor=south] at ($(0.25,0) + (-12,2)+0.5*(\x,0)$)  {\xintDecToHex{\x}};
        \foreach \x in {32,33,34,35,36,37,38,39}
            \node[anchor=south] at ($(0.25,0) + (-16,1.5)+0.5*(\x,0)$)  {\xintDecToHex{\x}};
        \foreach \x in {40,41,42,43,44,45,46,47}
            \node[anchor=south] at ($(0.25,0) + (-20,1.0)+0.5*(\x,0)$)  {\xintDecToHex{\x}};
        \foreach \x in {48,49,50,51,52,53,54,55}
            \node[anchor=south] at ($(0.25,0) + (-24,0.5)+0.5*(\x,0)$)  {\xintDecToHex{\x}};
        \foreach \x in {56,57,58,59,60,61,62,63}
            \node[anchor=south] at ($(0.25,0) + (-28,0.0)+0.5*(\x,0)$)  {\xintDecToHex{\x}};

    \end{pgfonlayer}

    % Foreground
    \begin{pgfonlayer}{foreground}
        
    \end{pgfonlayer} 

    % Background
    \begin{pgfonlayer}{background}
        % Grid
        \draw[step=0.5cm,gray] (0,0) grid (4,4);
    \end{pgfonlayer} 

\end{tikzpicture}
    \end{adjustbox}
    \caption{VHDL controller validation stimuli, hex values}
    \label{fig:vhdlcontrollerstimuli}
\end{figure}

The output stream that was stored in a text file by the VHDL
testbench was observed. The resulting pixel order was then validated using
random samples.
Listing \ref{lst:controlleroutstreamvhdl} shows the output pixel values of the
first three lines.

\begin{minipage}{\linewidth}
    \begin{lstlisting}[
        style=TextStyle, 
        caption=Output stream hexadecimal coded, 
        label=lst:controlleroutstreamvhdl
        ]
00 08 10 01 09 11 02 0A 12 03 0B 13 04 0C 14 05 0D 15 06 0E 16 07 0F 17 
08 10 18 09 11 19 0A 12 1A 0B 13 1B 0C 14 1C 0D 15 1D 0E 16 1E 0F 17 1F
10 18 20 11 19 21 12 1A 22 13 1B 23 14 1C 24 15 1D 25 16 1E 26 17 1F 27\end{lstlisting}
\end{minipage}

% ==============================================================================
%
%                             Overall Validation
%
% ==============================================================================
\subsection{Overall Validation}\label{ch:verification:overallvalidation}
To validate the overall system, the image processing and dataflow parts are
combined in a Vivado HLx project. Two projects are distinguished:
\begin{itemize}
    \item \texttt{diip}\footnote{\Gls{diip}} project using solution A) with 8bit HLS
    implementation
    \item \texttt{diip\_faster} project using solution B) with 8bit VHDL
    implementation
\end{itemize}

\pagebreak
On the computer side, the program \texttt{diip\_cc} written in C++ reads the
pixels from an input image, splits the data into according UFT packets and sends
them to the FPGA. The processed pixels are received and stored in the output
image. Similar to the proceeding in \ref{ch:verification:imageprocessing} two
test images were used and compared to the results of a floating point
calculation on the computer. The resulting \gls{rmse} is listed in table 
\ref{tab:overallvalidationresults}. The verification of the entire system on
the FPGA is positive. It can be seen that the HLS solution has the same
deviation as the simulation on the PC (see chapter 
\ref{ch:verification:imageprocessing}). The VHDL version has a smaller
deviation across the entire system than the \gls{ghdl} simulation on the PC (see
chapter \ref{ch:verification:imageprocessing}). It is assumed that the divider
model used in simulation is not as accurate as the implemented version of the
divider generator. Also shown here is that
the room image has a larger deviation in the VHDL solution than the
mountain image. This is to be justified with the same reasons as explained in
chapter \ref{ch:verification:imageprocessing}. It is assumed that this depends on the rounding and the image content. 


\begin{table}[h!] \centering \begin{tabular}{l c c}
        \toprule
        Solution            & RMSE room [\%]    & RMSE mountain [\%]\\
        \midrule
        diip         & 0.32             & 0.33 \\
        diip\_faster & 0.91             & 0.37 \\
        \bottomrule
    \end{tabular}
    \caption{Overall validation results}
    \label{tab:overallvalidationresults}
\end{table}


% ==============================================================================
%
%                             Benchmark
%
% ==============================================================================
\clearpage
\section{Benchmark}\label{ch:benchmark}
To conclude the results of this work, the different solutions are compared in
this chapter. The working implementations (including send and receive from PC)
and theoretical limits are differentiated. Furthermore, the throughput is
compared to a CPU based solution.
There are four different processing methods in total:

\begin{itemize}
    \item The solution using HLS with 8bit data bus
    \item The solution using HLS with 256bit data bus. This method was not
    implemented on the FPGA hence simulation results were used
    \item The solution implemented using VHDL 8bit
    \item A program running on CPU written in C++
\end{itemize}

Methods 1. and 3. are measured by processing an image stored on the computer
which is sent to the FPGA, processed and sent back. The time required for the
processing of an image divided by its pixel count results in the throughput
measured in megapixels per second ($Mp/s$, $1p\equiv1B$). To calculate the
throughput of
method 2. the results from \ref{ch:verification:imageprocessing} are used.
Method
4. is written in C++ and uses OpenCV to read image data. Throughput is
calculated the same as for methods 1. and 3.
\\

% \pagebreak
As a source image a fixed aspect ration of 16:9 was chosen. This is a common
format used among others for movies. The image was resized to four different
heights: 480p, 720p, 1080p and 2160p with the largest one being equivalent to
the
4k image standard. The content of the image does not influence the throughput.
\\

The benchmark is split into three separate comparisons. First the two overall
working solutions are compared (diip and diip\_faster). Then the theoretical
limits of the three image processing solution are compared without the influence
of a controller or the communication part. Finally the fastest image processing
solution is put in comparison with a CPU based processing.

\clearpage
\subsection{Compare Implementations}
First the two working FPGA implementations are put in contrast (1. and 3.).
Figure \ref{fig:benchmarkcompare} and table \ref{tab:throughputmeasuremetns}
show the results of the throughput measurements. The missing value for HLS 8bit
is due to a bug in the computer software for this image size that could not be
resolved by the time the benchmark was made. The HLS 8bit implementation yields
the lowest throughput. This is mainly due to the controller core which requires
the image data to be re-sent for every line the Wallis filter processes 
(no
image lines are cached on the FPGA) meaning
that the full potential of the HLS 8bit image processing core is not taken
advantage of. The VHDL implementation yields the
highest
throughput. This was to be expected for two
reasons. First, the VHDL Wallis filter core can process pixels at one pixel per
clock cycle which is more than the 8bit HLS version is capeable of (21 pixels
per 86 clock cycles).
Second, the memory management has been greatly improved from the HLS to the VHDL
controller in that image data is cached on the FPGA to reduce multiple
transmissions of image data from PC to FPGA. 

\begin{figure}[h!]
    \centering
    \begin{adjustbox}{max width=\linewidth}
        % \tikzsetnextfilename{system-overview}
\begin{tikzpicture}[
    rounded corners=0mm,
    triangle/.style = {fill=blue!20, regular polygon, regular polygon sides=3 },
    node rotated/.style = {rotate=180},
    border rotated/.style = {shape border rotate=180}
]
    %coordinates
    \coordinate (orig)      at (0,0);

    \begin{pgfonlayer}{main}
        \begin{axis}[
                % domain=-2:6,
                xtick={0,500,...,2500},
                ytick={0,1,...,18},
                xmajorgrids=true,ymajorgrids=true,
                xlabel={image height [$p$]},
                ylabel={throughput [$MB/s$]},
                title={Image Processing Benchmark},
                legend style={
                    cells={anchor=east},
                    legend pos=outer north east,
                },
                legend cell align=left,
                scatter/classes={
                    a={mark=triangle*,fill=black,draw=black},
                    b={mark=diamond*,fill=blue,draw=blue}
                    % b={mark=square*,fill=red,draw=red},
                    % d={mark=*,fill=brown,draw=brown}
                }
            ]
            \addplot[scatter,only marks, scatter src=explicit symbolic]
            table[meta=label] {
                x    y   label
                %% HLS 8 bit
                480  0.17 a
                % 720   a
                1080 0.168 a
                2160 0.161 a
                %% HLS 256bit
                % 480  11.3 b
                % 720  11.1 b
                % 1080 11.0 b
                % 2160 10.9 b
                %% VHDL
                480  0.89 b
                720  1.291 b
                1080 2.348 b
                2160 4.115 b
                %% CPU
                % 480  12.56 d
                % 720  16.23 d
                % 1080 15.99 d
                % 2160 15.91 d
            };
            \addlegendentry{HLS 8bit}
            % \addlegendentry{HLS 256bit}
            \addlegendentry{VHDL 8bit}
            % \addlegendentry{CPU}
        \end{axis}

    \end{pgfonlayer}

    % Foreground
    \begin{pgfonlayer}{foreground}
        
    \end{pgfonlayer} 

    % Background
    \begin{pgfonlayer}{background}
        % Grid
        % \draw[step=0.5cm,gray] (0,0) grid (4,4);
    \end{pgfonlayer} 

\end{tikzpicture}


% | Solution | Image | Throughput | Image File |
% |----------|-------|------------|------------|
% | HLS      | mountain | 0.168MB/s | mountain_fpga_hls.tif |
% | HLS      | room     | 0.170MB/s | room_fpga_hls.tif |
% | HLS      | cat480p  | 0.170MB/s |  |
% | HLS      | cat720p  | diip_cc error |  |
% | HLS      | cat1080p  | 0.168MB/s |  |
% | HLS      | cat2160p  | 0.161MB/s |  |
% | VHDL     | cat480p  | 0.89MB/s |  |
% | VHDL     | cat720p  | 1.291MB/s |  |
% | VHDL     | cat1080p  | 2.348MB/s |  |
% | VHDL     | cat2160p  | 4.115MB/s |  |
    \end{adjustbox}
    \caption{Wallis throughput comparison}
    \label{fig:benchmarkcompare}
\end{figure}

\begin{table}[h!]
    \centering
    \begin{tabular}{l l l l l l}
        \toprule
        Solution & 480p & 720p & 1080p & 2160p & \\
        \midrule
        HLS 8bit   & 0.17  &       & 0.168 & 0.161 & $MB/s$ \\
        % HLS 256bit & 11.3 & 11.1 & 11.0 & 10.9 \\
        VHDL 8bit      & 0.89  & 1.291 & 2.348 & 4.115 & $MB/s$\\
        % CPU        & 12.56 & 16.23 & 15.99 & 15.91 \\
        \bottomrule
    \end{tabular}
    \caption{Throughput measurements}
    \label{tab:throughputmeasuremetns}
\end{table}

\pagebreak
The increase in throughput over image size on the VHDL solution can be explained
by the way the
PC program works. To prevent an overflow of the image cache inside the
controller on the FPGA, a delay of $d_l$ seconds is inserted after an image line
is sent. This ensures that the FPGA is given enough time to calculate one image
line. From there the theoretical maximum throughput can be calculated. Equation
\ref{eq:theomaxb} shows the throughput (for derivation refer to appendix 
\ref{app:derivations:theomax}). $d_l$ was set to $500\mu s$ for the benchmark.

\begin{align}
    b  & \approx \frac{i_w}{d_l + \frac{i_w}{b_e}}
    \label{eq:theomaxb}
\end{align}

\begin{tabular}{rl}
    $b     =$ & theoretical throughput of VHDL solution \\
    $i_w   =$ & image width \\
    $d_l   =$ & delay between sending two image lines \\
    $b_e   =$ & ethernet throughput \\
\end{tabular} \\

If $d_l$ were to be reduced to zero, the throughput would theoretically strive
towards $b_e$. Although as explained in the next chapter, the maximum throughput
is limited by the Wallis filter core.

\subsection{Compare Theoretical Limits}
To show the real performance of an FPGA for image processing the theoretical
limits are put in contrast in this chapter without considering how the image
data is sent to the FPGA.
Before this can be done the maximum possible throughput of the three FPGA
implementations are calculated.

\begin{align}
    b_{ip}  & = \frac{i_hb_w}{w_l (i_h-w_l+1)}
    \label{eq:theomaxvhdlwallisshort}
\end{align}

\begin{tabular}{rl}
    $b_{ip}=$ & theoretical throughput image processing core \\
    $i_h   =$ & image height \\
    $b_w   =$ & throughput of image processing core on input in pixels per
    second \\
    $w_l   =$ & window length \\
\end{tabular} \\

Equation \ref{eq:theomaxvhdlwallisshort} is used to calculate the maximum
possible throughput for each of the test images (for derivation refer to
appendix \ref{app:derivations:theomaxvhdlwallis}). Figure \ref{fig:theoresults}
and table \ref{tab:FPGAimplementationstheoreticallimits} show the theoretical
maximum results.


\pagebreak
The HLS 8bit solution is the slowest of the three implementations. Its filter
core accepts 21 pixels per 86 clock cycles as described in table \ref{tab:throughput}. 
With the input width of the IP core extended to 265bit,
a new throughput of around 11 MB/s is achieved. The third solution written
in VHDL is placed in between the two HLS solutions. Its IP core accepts one
pixel per clock cycle. This yields a throughput of approximately 5.8MB/s at
125MHz clock frequency.

\begin{figure}[h!]
    \centering
    \begin{adjustbox}{max width=\linewidth}
        % \tikzsetnextfilename{system-overview}
\begin{tikzpicture}[
    rounded corners=0mm,
    triangle/.style = {fill=blue!20, regular polygon, regular polygon sides=3 },
    node rotated/.style = {rotate=180},
    border rotated/.style = {shape border rotate=180}
]
    %coordinates
    \coordinate (orig)      at (0,0);

    \begin{pgfonlayer}{main}
        \begin{axis}[
                % domain=-2:6,
                xtick={0,500,...,2500},
                ytick={0,2,...,18},
                ymin=0,
                xmajorgrids=true,ymajorgrids=true,
                xlabel={image height [$p$]},
                ylabel={throughput [$MB/s$]},
                title={Image Processing Theoretical Maximums},
                legend style={
                    cells={anchor=east},
                    legend pos=outer north east,
                },
                legend cell align=left,
                scatter/classes={
                    a={mark=triangle*,fill=black,draw=black},
                    b={mark=square*,fill=red,draw=red},
                    c={mark=diamond*,fill=blue,draw=blue}
                    % d={mark=*,fill=brown,draw=brown}
                }
            ]
            \addplot[scatter,only marks, scatter src=explicit symbolic]
            table[meta=label] {
                x    y   label
                %% HLS 8 bit
                480  1.45 a
                720  1.41 a
                1080 1.43 a
                2160 1.40 a
                %% HLS 256bit
                480  11.3 b
                720  11.1 b
                1080 11.0 b
                2160 10.9 b
                %% VHDL
                480  5.92 c
                720  5.84 c
                1080 5.78 c
                2160 5.73 c
                %% CPU
                % 480  12.56 d
                % 720  16.23 d
                % 1080 15.99 d
                % 2160 15.91 d
            };
            \addlegendentry{HLS 8bit}
            \addlegendentry{HLS 256bit}
            \addlegendentry{VHDL 8bit}
            % \addlegendentry{CPU}
        \end{axis}

    \end{pgfonlayer}

    % Foreground
    \begin{pgfonlayer}{foreground}
        
    \end{pgfonlayer} 

    % Background
    \begin{pgfonlayer}{background}
        % Grid
        % \draw[step=0.5cm,gray] (0,0) grid (4,4);
    \end{pgfonlayer} 

\end{tikzpicture}


% | Solution | Image | Throughput | Image File |
% |----------|-------|------------|------------|
% | HLS      | mountain | 0.168MB/s | mountain_fpga_hls.tif |
% | HLS      | room     | 0.170MB/s | room_fpga_hls.tif |
% | HLS      | cat480p  | 0.170MB/s |  |
% | HLS      | cat720p  | diip_cc error |  |
% | HLS      | cat1080p  | 0.168MB/s |  |
% | HLS      | cat2160p  | 0.161MB/s |  |
% | VHDL     | cat480p  | 0.89MB/s |  |
% | VHDL     | cat720p  | 1.291MB/s |  |
% | VHDL     | cat1080p  | 2.348MB/s |  |
% | VHDL     | cat2160p  | 4.115MB/s |  |
    \end{adjustbox}
    \caption{Theoretical maximum throughput of FPGA}
    \label{fig:theoresults}
\end{figure}

\begin{table}[h!]
    \centering
    \begin{tabular}{l l l l l l l}
        \toprule
        Solution & $b_w$ & 480p & 720p & 1080p & 2160p & \\
        \midrule
        HLS  8bit       & 29.1 & 1.45 & 1.43 & 1.41 & 1.40 & $MB/s$\\
        HLS  256bit     & 228  & 11.3 & 11.1 & 11.0 & 10.9 & $MB/s$\\
        VHDL 8bit          & 119  & 5.92 & 5.84 & 5.78 & 5.73 & $MB/s$\\
        \bottomrule
    \end{tabular}
    \caption{FPGA implementations theoretical limits}
    \label{tab:FPGAimplementationstheoreticallimits}
\end{table}

\clearpage
\subsection{FPGA against CPU}
Now that the different implementations are compared against each other, the
two implementations with the highest throughput (VHDL and HLS 256bit
implementations) are put in contrast with a CPU based computation.  The CPU
program was run multiple times for each image and the mean throughput was
calculated. The program was running on an Intel Core i7-6700HQ running at 2.60
GHz on a single core.  Figure \ref{fig:theoreticalmax} and table
\ref{tab:throughputcompare} show the results.
\\

The CPU performance is the fastest of all three. It processes the images 2.76
times faster than the VHDL solution and 1.45 times faster than the HLS 256bit
solution. As mentioned in the scalability chapter (\ref{chapt:scalability}) the
throughput of both FPGA implementations (VHDL and HLS 256bit) can be improved by
implementing the Wallis filter multiple times on one FPGA. By implementing the
HLS solution twice or the VHDL solution three times, a regular CPU could already
be outrunned by one FPGA.

\begin{figure}[h!]
    \centering
    \begin{adjustbox}{max width=\linewidth}
        % \tikzsetnextfilename{system-overview}
\begin{tikzpicture}[
    rounded corners=0mm,
    triangle/.style = {fill=blue!20, regular polygon, regular polygon sides=3 },
    node rotated/.style = {rotate=180},
    border rotated/.style = {shape border rotate=180}
]
    %coordinates
    \coordinate (orig)      at (0,0);

    \begin{pgfonlayer}{main}
        \begin{axis}[
                % domain=-2:6,
                xtick={0,500,...,2500},
                ymin=0,
                ytick={0,2,...,16},
                xmajorgrids=true,ymajorgrids=true,
                xlabel={image height [$p$]},
                ylabel={throughput [$MB/s$]},
                title={Theoretical Maximum vs CPU},
                legend style={
                    cells={anchor=east},
                    legend pos=outer north east,
                },
                legend cell align=left,
                scatter/classes={
                    a={mark=square*,fill=red,draw=red},
                    b={mark=diamond*,fill=blue,draw=blue},
                    c={mark=*,fill=brown,draw=brown}
                }
            ]
            \addplot[scatter,only marks, scatter src=explicit symbolic]
            table[meta=label] {
                x    y   label
                %% HLS 8 bit
                % 480  1.45 a
                % 720  1.41 a
                % 1080 1.43 a
                % 2160 1.40 a
                %% HLS 256bit
                480  11.3 a
                720  11.1 a
                1080 11.0 a
                2160 10.9 a
                %% VHDL
                480  5.92 b
                720  5.84 b
                1080 5.78 b
                2160 5.73 b
                %% CPU
                480  12.56 c
                720  16.23 c
                1080 15.99 c
                2160 15.91 c
            };
            % \addlegendentry{FPGA theoretical max}
            % \addlegendentry{HLS 8bit}
            \addlegendentry{HLS 256bit}
            \addlegendentry{VHDL}
            \addlegendentry{CPU}
        \end{axis}

    \end{pgfonlayer}

    % Foreground
    \begin{pgfonlayer}{foreground}
        
    \end{pgfonlayer} 

    % Background
    \begin{pgfonlayer}{background}
        % Grid
        % \draw[step=0.5cm,gray] (0,0) grid (4,4);
    \end{pgfonlayer} 

\end{tikzpicture}


% | Solution | Image | Throughput | Image File |
% |----------|-------|------------|------------|
% | HLS      | mountain | 0.168MB/s | mountain_fpga_hls.tif |
% | HLS      | room     | 0.170MB/s | room_fpga_hls.tif |
% | HLS      | cat480p  | 0.170MB/s |  |
% | HLS      | cat720p  | diip_cc error |  |
% | HLS      | cat1080p  | 0.168MB/s |  |
% | HLS      | cat2160p  | 0.161MB/s |  |
% | VHDL     | cat480p  | 0.89MB/s |  |
% | VHDL     | cat720p  | 1.291MB/s |  |
% | VHDL     | cat1080p  | 2.348MB/s |  |
% | VHDL     | cat2160p  | 4.115MB/s |  |
    \end{adjustbox}
    \caption{Theoretical maximum throughput of FPGA versus CPU}
    \label{fig:theoreticalmax}
\end{figure}

\begin{table}[h!]
    \centering
    \begin{tabular}{l l l l l l}
        \toprule
        Solution & 480p & 720p & 1080p & 2160p & \\
        \midrule
        VHDL 8bit       & 5.92  & 5.84  & 5.78  & 5.73  & $MB/s$ \\
        HLS  256bit & 11.3  & 11.1  & 11.0  & 10.9  & $MB/s$ \\
        CPU        & 12.56 & 16.23 & 15.99 & 15.91 & $MB/s$ \\
        \bottomrule
    \end{tabular}
    \caption{FPGA PC Throughput comparison}
    \label{tab:throughputcompare}
\end{table}

\subsection{Throughput against Area Efficiency}
Finally, the throughput efficiency of the Wallis filter is examined. The three
Wallis filter implementations are compared. The efficiency of the
implementations is examined depending on the area they use on the FPGA
and their theoretical throughput. The area is shown in table 
\ref{tab:ressource}. Only one resource is considered for meaningful efficiency.
The DSP is omitted for comparison, since all three implementations consume the
same amount. The block memory is also not selected, because the controller needs
much more BRAM than the Wallis filter and it needs less memory than slices in
percentage. For this reason, the slice usage is compared for throughput
efficiency. Table \ref{tab:efficiency} lists the comparison. The efficiency is
measured in kB/s per slice.
\\

The VHDL code yields the best efficiency by far. This can be explained in two
reasons: One, the VHDL code was written on register transfer level, meaning that
every register and combinatorial logic was placed deliberate while the HLS
synthesis tool optimizes C/C++ code to a certain degree but not perfect. This
can be compared to writing microcontroller code in either C or assembly
language. The second reason being that the dataflow inside the core could be
described more precise in VHDL leading to a higher throughput.
\\

The difference in the two HLS implementations can be explained by the way the
256bit differs from the 8bit version. The increase in input width resulted only
in a minor increase in resource usage. But it enabled the whole processing to be
pipelined which requires little more resources but yields high throughput
increase. The strongly increased throughput lead to an efficiency increase of a
factor of five.
\\

\begin{table}[h!]
    \centering
    \begin{tabular}{l l l l}
        \toprule
        DUV         & Slice usage & Throughput [MB/s] & Efficiency [kB/s per
        slice]\\
        \midrule
        HLS  8bit    &  969          & 1.4               & 1.5   \\
        HLS  256bit  &  1542         & 11.0              & 7.3   \\
        VHDL 8bit    &  253          & 5.8               & 23.5  \\
        \bottomrule
    \end{tabular}
    \caption{Efficiency of the Wallis filter in relation to area and throughput}
    \label{tab:efficiency}
\end{table}






