% ==============================================================================
%
%                             Verification & Benchmark
%
% ==============================================================================
\chapter{Verification \& Benchmark} \label{chapt:ver_bench}
With the image processing an dataflow parts implemented, they can be verified
and benchmarked. The next chapters hold the verification process of both the
image processing part and the dataflow part. After both components are verified
they are benchmarked as one unity against a computer based implementation in
chapter \ref{ch:benchmark}.


% ==============================================================================
%
%                             Verification
%
% ==============================================================================
\section{Verification} \label{ch:verification}
The verification process ensures that all components work as expected. It is
split into the image processing and dataflow parts. They were tested
independantly to reduce complexity and simulation time. The system as a unity is
then tested in chapter \ref{ch:benchmark} benchmark.

% ==============================================================================
%
%                             Image Processing
%
% ==============================================================================
\subsection{Image Processing}\label{ch:verification:imageprocessing}

% ==============================================================================
%
%                             Dataflow
%
% ==============================================================================
\subsection{Dataflow}\label{ch:verification:dataflow}
The dataflow part is again divided into two parts, the communication and
controller parts. While in the communication part only the most recent version
is tested (reffering to chapter \ref{chapt:dataflow} explaining the solotion
A and B with streaming interface), both versions of the controller are verified.

\subsubsection*{Communication}
The communication part was in a large part taken from the last semester project
and was been thoroughly tested and validated in the project report 
\cite{p5report}. The three new implemented features are verified in this
chapter. They consist of:
\begin{itemize}
    \item Acknowledge
    \item User registers
    \item Stream interface
\end{itemize}

\vspace{1ex}
\textbf{Acknowledge:} To test the acknowledge function, a file was sent to the
FPGA from the computer
and the Ethernet traffic monitored with Wireshark Netowrk Protocol Analyzer.
Wiresharks packet dissections are exported to a json file that was then analyzed
using the \texttt{uftcheck} utility. It was written to analyze network pacekts
for UFT transfers. The first lines of output yielded:

\vspace{1ex}
    % \begin{adjustbox}{max width=1.7\textwidth}
\begin{minipage}{1\linewidth}
    \begin{lstlisting}[
        style=TextStyle, 
        % caption=ack buffer allocation, 
        % label=lst:ackbufalloc
        ]
+------+----------------------+----------------------+-----+------------+------+------+
| Pack |         From         |          To          | D/C |  Control   | TCID | SEQ  | 
+------+----------------------+----------------------+-----+------------+------+------+
|  1   | 192.168.5.10 (50719) | 192.168.5.9 (42042)  |  C  |  FT Start  |  12  | 1036 |
|  2   | 192.168.5.10 (50719) | 192.168.5.9 (42042)  |  D  |            |  12  |  0   |
|  3   | 192.168.5.9 (42042)  | 192.168.5.10 (50719) |  C  | ACK packet |  12  |  0   |
|  5   | 192.168.5.10 (50719) | 192.168.5.9 (42042)  |  D  |            |  12  |  1   |
|  6   | 192.168.5.9 (42042)  | 192.168.5.10 (50719) |  C  | ACK packet |  12  |  1   |
\end{lstlisting}
\end{minipage}
% \end{adjustbox}

\vspace{1ex}
The sending PC starts a file transmission with a file transfer start packet and
the first data packet. The next packet (3) is comming from the FPGA to the PC
and acknowledges the first data packet (sequence 0). To verify that all packets
are acknowledged, the sending program reports an acknowledge status after the
files has been sent. 

\begin{minipage}{1\linewidth}
    \begin{lstlisting}[
        style=TextStyle, 
        % caption=ack buffer allocation, 
        % label=lst:ackbufalloc
        ]
$ ./sender 192.168.5.9 42042 payload/cat.jpg
UFT Sender demo
destination 192.168.5.9:42042
HURRAY! All 1036 packets have been acknowledged.
time elapsed: 1.18s Speed: 0.859 MB/s Size: 1.012 MB\end{lstlisting}
\end{minipage}

\vspace{1ex}
\textbf{User register:} To test wether the user registers can be written and are
output correctly by the communication core, a \gls{ila} 
was used. An ILA can be configured on the FPGA to record internal signals. The
results are transferred to the PC over USB and displayed in \gls{vivadohlx}. All
user registers (0 through 7) are written with differen values and the reult
observed using the ILA matched the sent data.

\vspace{1ex}
\textbf{Stream interface:} The last modification made to the communication core
was the AXI4-Stream interface. This was similarly tested as the user registers.
Using an ILA the output of the stream was observed and the correct order of data
verified. 

\subsubsection*{Controller} 
The most important thing to verify in the controller core is the correct order
of output pixel. This is done by generating input data representing an image,
feeding this data through the controller core and observing the output pixels.
This validation is split into the two solutions implemented using HLS (solution
A) and VHDL (solution B).

\vspace{1ex}
\textbf{Solution A) HLS:} In Vivado HLS the validation was done in C and
Co-simulation. A testbench C/C++ file generates pseudo random image data. This
data is then processed using the controller core and in the testbench itself.
The two results are then compared. Using different input data and image sizes,
all output results matched. A test on the FPGA was omitted. This validation will
later take plave in the overall validation in chapter \ref{ch:verification:overallvalidation}.

\vspace{1ex}
\textbf{Solution B) VHDL:} The same approach was used in the solution written in
VHDL except that the testbench is not written in C/C++ but in VHDL. The input
stream sends an incrementing pixel value with an image width and height of
eight. The window length was reduced to three for simplicity. Figure 
\ref{fig:vhdlcontrollerstimuli} shows the input data values.

 \begin{figure}[h!]
    \centering
    \begin{adjustbox}{max width=\linewidth}
        % \tikzsetnextfilename{system-overview}
\begin{tikzpicture}[
    rounded corners=0mm,
    triangle/.style = {fill=blue!20, regular polygon, regular polygon sides=3 },
    node rotated/.style = {rotate=180},
    border rotated/.style = {shape border rotate=180}
]
    %coordinates
    \coordinate (orig)      at (0,0);

    \begin{pgfonlayer}{main}
        
        % Text
        % \node[] (write) at (-2,5) {Write};
        % \node[] (read) at (0,6.2) {Read};

        % Braces
        \draw [line width=0.5mm,decorate,decoration={brace,amplitude=10pt},xshift=-4pt,yshift=0pt] (4.5,4) -- (4.5,0) node [black,midway,xshift=0.5cm,anchor=west] {Height};
        \draw [line width=0.5mm,decorate,decoration={brace,amplitude=10pt},xshift=-0pt,yshift=4pt] (4,-0.5) -- (0,-0.5) node [black,midway,yshift=-0.5cm,anchor=north] {Width};
        
        % Center pixel
        \draw[black,line width=0.5mm] (0.5,3.0) rectangle (1,3.5);
        % Window size
        \draw[black,line width=0.5mm] (0,2.5) rectangle (1.5,4.0);

        % Axis
        \foreach \x in {0,1,2,3,4,5,6,7}
            \node[anchor=south] at ($(0.25,0) + (0,3.5)+0.5*(\x,0)$)  {0\padzeroes[1]\Hexadecimalnum{\x}};
        \foreach \x in {8,9,10,11,12,13,14,15}
            \node[anchor=south] at ($(0.25,0) + (-4,3.0)+0.5*(\x,0)$)  {0\padzeroes[1]\Hexadecimalnum{\x}};
        \foreach \x in {16,17,18,19,20,21,22,23}
            \node[anchor=south] at ($(0.25,0) + (-8,2.5)+0.5*(\x,0)$)  {\xintDecToHex{\x}};
        \foreach \x in {24,25,26,27,28,29,30,31}
            \node[anchor=south] at ($(0.25,0) + (-12,2)+0.5*(\x,0)$)  {\xintDecToHex{\x}};
        \foreach \x in {32,33,34,35,36,37,38,39}
            \node[anchor=south] at ($(0.25,0) + (-16,1.5)+0.5*(\x,0)$)  {\xintDecToHex{\x}};
        \foreach \x in {40,41,42,43,44,45,46,47}
            \node[anchor=south] at ($(0.25,0) + (-20,1.0)+0.5*(\x,0)$)  {\xintDecToHex{\x}};
        \foreach \x in {48,49,50,51,52,53,54,55}
            \node[anchor=south] at ($(0.25,0) + (-24,0.5)+0.5*(\x,0)$)  {\xintDecToHex{\x}};
        \foreach \x in {56,57,58,59,60,61,62,63}
            \node[anchor=south] at ($(0.25,0) + (-28,0.0)+0.5*(\x,0)$)  {\xintDecToHex{\x}};

    \end{pgfonlayer}

    % Foreground
    \begin{pgfonlayer}{foreground}
        
    \end{pgfonlayer} 

    % Background
    \begin{pgfonlayer}{background}
        % Grid
        \draw[step=0.5cm,gray] (0,0) grid (4,4);
    \end{pgfonlayer} 

\end{tikzpicture}
    \end{adjustbox}
    \caption{VHDL controller validation stimuli}
    \label{fig:vhdlcontrollerstimuli}
\end{figure}

Observed was the output stream that was stored in a text file by the VHDL
testbench. The resulting pixel order was then validated using random samples.
Listing \ref{lst:controlleroutstreamvhdl} shows the output pixel values of the
first three lines.

\begin{minipage}{\linewidth}
    \begin{lstlisting}[
        style=TextStyle, 
        caption=Output stream hexadecimal coded, 
        label=lst:controlleroutstreamvhdl
        ]
00 08 10 01 09 11 02 0A 12 03 0B 13 04 0C 14 05 0D 15 06 0E 16 07 0F 17 
08 10 18 09 11 19 0A 12 1A 0B 13 1B 0C 14 1C 0D 15 1D 0E 16 1E 0F 17 1F
10 18 20 11 19 21 12 1A 22 13 1B 23 14 1C 24 15 1D 25 16 1E 26 17 1F 27\end{lstlisting}
\end{minipage}

% ==============================================================================
%
%                             Overall Validation
%
% ==============================================================================
\subsection{Overall Validation}\label{ch:verification:overallvalidation}
To validate the overall system, the image processing and dataflow parts are
combined in a Vivado HLx project. Two projects are distinguished:
\begin{itemize}
    \item \texttt{diip}\footnote{\Gls{diip}} Project using solution A) with HLS
    implementation
    \item \texttt{diip\_faster} Project using solution B) with VHDL
    implementation
\end{itemize}

On the computer side, the program \texttt{diip\_cc} written in C++ reads the
pixels from an input image, splits the data into according UFT packets and sends
them to the FPGA. The processed pixels are received and stored in the output
image. Similar to the proceeding in \ref{ch:verification:imageprocessing} two
test images were used and compared to the results of a floating point
calculation on the computer. The resulting \gls{rmse} is listed in table \ref{tab:overallvalidationresults}. 

\todo[inline]{Fill table}
\begin{table}[tb!]
    \centering
    \begin{tabular}{l c c}
        \toprule
        Solution & RMSE room & RMSE mountain \\
        \midrule
        diip (HLS) & 0.001 & 0.001 \\
        diip\_faster (VHDL) & 0.001 & 0.001 \\
        \bottomrule
    \end{tabular}
    \caption{Overall validation results}
    \label{tab:overallvalidationresults}
\end{table}


% ==============================================================================
%
%                             Benchmark
%
% ==============================================================================
\section{Benchmark}\label{ch:benchmark}
To conclude the results of this work, the different solutions are compared in
this chapter. The working implementations (including send and receive from PC)
and theoretical limits are differentiated. Furthermore the throughput is
compared to a CPU based solution.
There are four different processing methods in total.

\begin{itemize}
    \item The solution using HLS with 8bit data bus
    \item The solution using HLS with 256bit data bus. This method was not
    implemented on the FPGA hence simulation results were used
    \item The solution implemented using VHDL
    \item A programm running on CPU written in C++
\end{itemize}

Methods 1. and 3. are measured by processing an image stored on the computer
which is sent to the FPGA, processed and sent back. The time required for the
processing of an image divided by its pixel count result in the throughput
measured in megapixels per second ($Mp/s$). To calculate the throughput of
method 2. the results from \ref{ch:verification:imageprocessing} is used. Method
4. is written in C++ and uses OpenCV to read image data. Throughput is
calculated the same as for methods 1. and 3.
\\

As source image a fixed aspect ration of 16:9 was chosen. This is a common
format used among other for movies. The image was resized to four different
heights: 480p, 720p, 1080p and 2160p with the largest one being equivalent to
the
4k image standard. The content of the image does not influence the throughput.

\subsection{Compare implementations}
First the two working FPGA implementations are put to contrast (1. and 3.).
Figure \ref{fig:benchmarkcompare} and table \ref{tab:throughputmeasuremetns}
show the results of the throughput measurements. The missing value for HLS 8bit
is due to a bug in the computer software for this image size that could not be
resolved by the time the benchmark was made. The HLS 8bit implementation yields
the lowest throughput. This is mainly due to the controller core which requires
the image data to be re-sent for every line the wallis filter processes. No
image lines are buffered on the FPGA. The VHDL implementation yields the
highest
throughput. This was to be expected for two
reasons. First, the VHDL wallis filter core can process pixels at one pixel per
clock cycle which is more than the 8bit HLS version is capeable of (21 pixels
per 86 clock cycles).
Second, the memory management was been greatly improved from the HLS to the VHDL
controller in that image data is buffered on the FPGA to reduce multiple
transmissions of image data from PC to FPGA. 

\begin{figure}[tb!]
    \centering
    \begin{adjustbox}{max width=\linewidth}
        % \tikzsetnextfilename{system-overview}
\begin{tikzpicture}[
    rounded corners=0mm,
    triangle/.style = {fill=blue!20, regular polygon, regular polygon sides=3 },
    node rotated/.style = {rotate=180},
    border rotated/.style = {shape border rotate=180}
]
    %coordinates
    \coordinate (orig)      at (0,0);

    \begin{pgfonlayer}{main}
        \begin{axis}[
                % domain=-2:6,
                xtick={0,500,...,2500},
                ytick={0,1,...,18},
                xmajorgrids=true,ymajorgrids=true,
                xlabel={image height [$p$]},
                ylabel={throughput [$MB/s$]},
                title={Image Processing Benchmark},
                legend style={
                    cells={anchor=east},
                    legend pos=outer north east,
                },
                legend cell align=left,
                scatter/classes={
                    a={mark=triangle*,fill=black,draw=black},
                    b={mark=diamond*,fill=blue,draw=blue}
                    % b={mark=square*,fill=red,draw=red},
                    % d={mark=*,fill=brown,draw=brown}
                }
            ]
            \addplot[scatter,only marks, scatter src=explicit symbolic]
            table[meta=label] {
                x    y   label
                %% HLS 8 bit
                480  0.17 a
                % 720   a
                1080 0.168 a
                2160 0.161 a
                %% HLS 256bit
                % 480  11.3 b
                % 720  11.1 b
                % 1080 11.0 b
                % 2160 10.9 b
                %% VHDL
                480  0.89 b
                720  1.291 b
                1080 2.348 b
                2160 4.115 b
                %% CPU
                % 480  12.56 d
                % 720  16.23 d
                % 1080 15.99 d
                % 2160 15.91 d
            };
            \addlegendentry{HLS 8bit}
            % \addlegendentry{HLS 256bit}
            \addlegendentry{VHDL 8bit}
            % \addlegendentry{CPU}
        \end{axis}

    \end{pgfonlayer}

    % Foreground
    \begin{pgfonlayer}{foreground}
        
    \end{pgfonlayer} 

    % Background
    \begin{pgfonlayer}{background}
        % Grid
        % \draw[step=0.5cm,gray] (0,0) grid (4,4);
    \end{pgfonlayer} 

\end{tikzpicture}


% | Solution | Image | Throughput | Image File |
% |----------|-------|------------|------------|
% | HLS      | mountain | 0.168MB/s | mountain_fpga_hls.tif |
% | HLS      | room     | 0.170MB/s | room_fpga_hls.tif |
% | HLS      | cat480p  | 0.170MB/s |  |
% | HLS      | cat720p  | diip_cc error |  |
% | HLS      | cat1080p  | 0.168MB/s |  |
% | HLS      | cat2160p  | 0.161MB/s |  |
% | VHDL     | cat480p  | 0.89MB/s |  |
% | VHDL     | cat720p  | 1.291MB/s |  |
% | VHDL     | cat1080p  | 2.348MB/s |  |
% | VHDL     | cat2160p  | 4.115MB/s |  |
    \end{adjustbox}
    \caption{Wallis throughput comparisson}
    \label{fig:benchmarkcompare}
\end{figure}

\begin{table}[tb!]
    \centering
    \begin{tabular}{l l l l l l}
        \toprule
        Solution & 480p & 720p & 1080p & 2160p & \\
        \midrule
        HLS 8bit   & 0.17  &       & 0.168 & 0.161 & $MB/s$ \\
        % HLS 256bit & 11.3 & 11.1 & 11.0 & 10.9 \\
        VHDL       & 0.89  & 1.291 & 2.348 & 4.115 & $MB/s$\\
        % CPU        & 12.56 & 16.23 & 15.99 & 15.91 \\
        \bottomrule
    \end{tabular}
    \caption{Throughput measurements}
    \label{tab:throughputmeasuremetns}
\end{table}

The increase in throughput over image size on the VHDL solution can be explained
by the way the
PC program works. To prevent an overflow of the image cache inside the
controller on the FPGA, a delay of $d_l$ seconds is inserted after an image line
is sent. This ensures that the FPGA is given enough time to calculate one image
line. From there the theoretical maximum throughput can be calculated. Equation
\ref{eq:theomaxb} shows the throughput (for derivation refer to appendix 
\ref{app:derivations:theomax}). $d_l$ was set to $500\mu s$ for the benchmark.

\begin{align}
    b  & \approx \frac{i_w}{d_l + \frac{i_w}{b_e}}
    \label{eq:theomaxb}
\end{align}

\begin{tabular}{rl}
    $b     =$ & theoretical throughput of VHDL solution \\
    $i_w   =$ & image width \\
    $d_l   =$ & delay between sending two image lines \\
    $b_e   =$ & ethernet throughput \\
\end{tabular} \\

If $d_l$ were to be reduced to zero, the throughput would strive towards $b_e$.

\subsection{Compare theoretical limits}
To show the real performance of a FPGA for image processing the theoretical
limits are put to contrast in this chapter without considering how the image
data is sent to the FPGA.
Before this can be done the maximum possible throughput of the three FPGA
implementations are calculated.

\begin{align}
    b_{ip}  & = \frac{i_hb_w}{w_l (i_h-w_l+1)}
    \label{eq:theomaxvhdlwallisshort}
\end{align}

\begin{tabular}{rl}
    $b_{ip}=$ & theoretical throughput image processing core \\
    $i_h   =$ & image height \\
    $b_w   =$ & throughput of image processing core on input in pixels per
    second \\
    $w_l   =$ & window length \\
\end{tabular} \\

Equation \ref{eq:theomaxvhdlwallisshort} is used to calculate the maximum
possible throughput for each of the test images (for derivation refer to
appendix \ref{app:derivations:theomaxvhdlwallis}). Figure \ref{fig:theoresults}
and table \ref{tab:FPGAimplementationstheoreticallimits} show the theoretical
maximum results.

\begin{figure}[b!]
    \centering
    \begin{adjustbox}{max width=\linewidth}
        % \tikzsetnextfilename{system-overview}
\begin{tikzpicture}[
    rounded corners=0mm,
    triangle/.style = {fill=blue!20, regular polygon, regular polygon sides=3 },
    node rotated/.style = {rotate=180},
    border rotated/.style = {shape border rotate=180}
]
    %coordinates
    \coordinate (orig)      at (0,0);

    \begin{pgfonlayer}{main}
        \begin{axis}[
                % domain=-2:6,
                xtick={0,500,...,2500},
                ytick={0,2,...,18},
                ymin=0,
                xmajorgrids=true,ymajorgrids=true,
                xlabel={image height [$p$]},
                ylabel={throughput [$MB/s$]},
                title={Image Processing Theoretical Maximums},
                legend style={
                    cells={anchor=east},
                    legend pos=outer north east,
                },
                legend cell align=left,
                scatter/classes={
                    a={mark=triangle*,fill=black,draw=black},
                    b={mark=square*,fill=red,draw=red},
                    c={mark=diamond*,fill=blue,draw=blue}
                    % d={mark=*,fill=brown,draw=brown}
                }
            ]
            \addplot[scatter,only marks, scatter src=explicit symbolic]
            table[meta=label] {
                x    y   label
                %% HLS 8 bit
                480  1.45 a
                720  1.41 a
                1080 1.43 a
                2160 1.40 a
                %% HLS 256bit
                480  11.3 b
                720  11.1 b
                1080 11.0 b
                2160 10.9 b
                %% VHDL
                480  5.92 c
                720  5.84 c
                1080 5.78 c
                2160 5.73 c
                %% CPU
                % 480  12.56 d
                % 720  16.23 d
                % 1080 15.99 d
                % 2160 15.91 d
            };
            \addlegendentry{HLS 8bit}
            \addlegendentry{HLS 256bit}
            \addlegendentry{VHDL 8bit}
            % \addlegendentry{CPU}
        \end{axis}

    \end{pgfonlayer}

    % Foreground
    \begin{pgfonlayer}{foreground}
        
    \end{pgfonlayer} 

    % Background
    \begin{pgfonlayer}{background}
        % Grid
        % \draw[step=0.5cm,gray] (0,0) grid (4,4);
    \end{pgfonlayer} 

\end{tikzpicture}


% | Solution | Image | Throughput | Image File |
% |----------|-------|------------|------------|
% | HLS      | mountain | 0.168MB/s | mountain_fpga_hls.tif |
% | HLS      | room     | 0.170MB/s | room_fpga_hls.tif |
% | HLS      | cat480p  | 0.170MB/s |  |
% | HLS      | cat720p  | diip_cc error |  |
% | HLS      | cat1080p  | 0.168MB/s |  |
% | HLS      | cat2160p  | 0.161MB/s |  |
% | VHDL     | cat480p  | 0.89MB/s |  |
% | VHDL     | cat720p  | 1.291MB/s |  |
% | VHDL     | cat1080p  | 2.348MB/s |  |
% | VHDL     | cat2160p  | 4.115MB/s |  |
    \end{adjustbox}
    \caption{Theoretical maximum throughput of FPGA}
    \label{fig:theoresults}
\end{figure}

\begin{table}[b!]
    \centering
    \begin{tabular}{l l l l l l l}
        \toprule
        Solution & $b_w$ & 480p & 720p & 1080p & 2160p & \\
        \midrule
        HLS 8bit       & 29.1 & 1.45 & 1.43 & 1.41 & 1.40 & $MB/s$\\
        HLS 256bit     & 228  & 11.3 & 11.1 & 11.0 & 10.9 & $MB/s$\\
        VHDL           & 119  & 5.92 & 5.84 & 5.78 & 5.73 & $MB/s$\\
        \bottomrule
    \end{tabular}
    \caption{FPGA implementations theoretical limits}
    \label{tab:FPGAimplementationstheoreticallimits}
\end{table}

The HLS 8bit solution is the slowest of the three implementations. Its filter
core accepts 21 pixels per 86 clock cycles as described in \todo{ref
to hls8bit validation}. With the input width of the IP-core extended to 265bit,
a new throughput of around 11 MB/s is achieved. The third solution written
in VHDL is placed in between the two HLS solutions. Its IP-core accepts one
pixel per clock cycle. This yields a throughput of approximately 5.8MB/s at
125MHz clock frequency.

\clearpage
\subsection{FPGA versus CPU}
Now that the different implementations are compared against each other, the
two implementations with the highest throughput (VHDL and HLS 256bit
implementations) are put to contrast with a CPU based computation.  The CPU
programm was run multiple times for each image and the mean throughput is
calculated. Figure \ref{fig:theoreticalmax} and table 
\ref{tab:throughputcompare} show the results.

\begin{figure}[b!]
    \centering
    \begin{adjustbox}{max width=\linewidth}
        % \tikzsetnextfilename{system-overview}
\begin{tikzpicture}[
    rounded corners=0mm,
    triangle/.style = {fill=blue!20, regular polygon, regular polygon sides=3 },
    node rotated/.style = {rotate=180},
    border rotated/.style = {shape border rotate=180}
]
    %coordinates
    \coordinate (orig)      at (0,0);

    \begin{pgfonlayer}{main}
        \begin{axis}[
                % domain=-2:6,
                xtick={0,500,...,2500},
                ymin=0,
                ytick={0,2,...,16},
                xmajorgrids=true,ymajorgrids=true,
                xlabel={image height [$p$]},
                ylabel={throughput [$MB/s$]},
                title={Theoretical Maximum vs CPU},
                legend style={
                    cells={anchor=east},
                    legend pos=outer north east,
                },
                legend cell align=left,
                scatter/classes={
                    a={mark=square*,fill=red,draw=red},
                    b={mark=diamond*,fill=blue,draw=blue},
                    c={mark=*,fill=brown,draw=brown}
                }
            ]
            \addplot[scatter,only marks, scatter src=explicit symbolic]
            table[meta=label] {
                x    y   label
                %% HLS 8 bit
                % 480  1.45 a
                % 720  1.41 a
                % 1080 1.43 a
                % 2160 1.40 a
                %% HLS 256bit
                480  11.3 a
                720  11.1 a
                1080 11.0 a
                2160 10.9 a
                %% VHDL
                480  5.92 b
                720  5.84 b
                1080 5.78 b
                2160 5.73 b
                %% CPU
                480  12.56 c
                720  16.23 c
                1080 15.99 c
                2160 15.91 c
            };
            % \addlegendentry{FPGA theoretical max}
            % \addlegendentry{HLS 8bit}
            \addlegendentry{HLS 256bit}
            \addlegendentry{VHDL}
            \addlegendentry{CPU}
        \end{axis}

    \end{pgfonlayer}

    % Foreground
    \begin{pgfonlayer}{foreground}
        
    \end{pgfonlayer} 

    % Background
    \begin{pgfonlayer}{background}
        % Grid
        % \draw[step=0.5cm,gray] (0,0) grid (4,4);
    \end{pgfonlayer} 

\end{tikzpicture}


% | Solution | Image | Throughput | Image File |
% |----------|-------|------------|------------|
% | HLS      | mountain | 0.168MB/s | mountain_fpga_hls.tif |
% | HLS      | room     | 0.170MB/s | room_fpga_hls.tif |
% | HLS      | cat480p  | 0.170MB/s |  |
% | HLS      | cat720p  | diip_cc error |  |
% | HLS      | cat1080p  | 0.168MB/s |  |
% | HLS      | cat2160p  | 0.161MB/s |  |
% | VHDL     | cat480p  | 0.89MB/s |  |
% | VHDL     | cat720p  | 1.291MB/s |  |
% | VHDL     | cat1080p  | 2.348MB/s |  |
% | VHDL     | cat2160p  | 4.115MB/s |  |
    \end{adjustbox}
    \caption{Theoretical maximum throughput of FPGA versus CPU}
    \label{fig:theoreticalmax}
\end{figure}

\begin{table}[b!]
    \centering
    \begin{tabular}{l l l l l l}
        \toprule
        Solution & 480p & 720p & 1080p & 2160p & \\
        \midrule
        VHDL       & 5.92  & 5.84  & 5.78  & 5.73  & $MB/s$\\
        HLS 256bit & 11.3  & 11.1  & 11.0  & 10.9  & $MB/s$\\
        CPU        & 12.56 & 16.23 & 15.99 & 15.91 & $MB/s$\\
        \bottomrule
    \end{tabular}
    \caption{FPGA PC Throughput comparisson}
    \label{tab:throughputcompare}
\end{table}



