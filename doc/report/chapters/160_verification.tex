% ==============================================================================
%
%                             Verification & Benchmark
%
% ==============================================================================
\chapter{Verification \& Benchmark} \label{chapt:ver_bench}
With the image processing an dataflow parts implemented, they can be verified
and benchmarked. The next chapters hold the verification process of both the
image processing part and the dataflow part. After both components are verified
they are benchmarked as one unity against a computer based implementation in
chapter \ref{ch:benchmark}.


% ==============================================================================
%
%                             Verification
%
% ==============================================================================
\section{Verification} \label{ch:verification}
The verification process ensures that all components work as expected. It is
split into the image processing and dataflow parts. They were tested
independantly to reduce complexity and simulation time. The system as a unity is
then tested in chapter \ref{ch:benchmark} benchmark.

% ==============================================================================
%
%                             Image Processing
%
% ==============================================================================
\subsection{Image Processing}\label{ch:verification:imageprocessing}

% ==============================================================================
%
%                             Dataflow
%
% ==============================================================================
\subsection{Dataflow}\label{ch:verification:dataflow}
The dataflow part is again divided into two parts, the communication and
controller parts. While in the communication part only the most recent version
is tested (reffering to chapter \ref{chapt:dataflow} explaining the solotion
A and B with streaming interface), both versions of the controller are verified.

\subsubsection*{Communication}
The communication part was in a large part taken from the last semester project
and was been thoroughly tested and validated in the project report 
\cite{p5report}. The three new implemented features are verified in this
chapter. They consist of:
\begin{itemize}
    \item Acknowledge
    \item User registers
    \item Stream interface
\end{itemize}

\vspace{1ex}
\textbf{Acknowledge:} To test the acknowledge function, a file was sent to the
FPGA from the computer
and the Ethernet traffic monitored with Wireshark Netowrk Protocol Analyzer.
Wiresharks packet dissections are exported to a json file that was then analyzed
using the \texttt{uftcheck} utility. It was written to analyze network pacekts
for UFT transfers. The first lines of output yielded:

\vspace{1ex}
    % \begin{adjustbox}{max width=1.7\textwidth}
\begin{minipage}{1\linewidth}
    \begin{lstlisting}[
        style=TextStyle, 
        % caption=ack buffer allocation, 
        % label=lst:ackbufalloc
        ]
+------+----------------------+----------------------+-----+------------+------+------+
| Pack |         From         |          To          | D/C |  Control   | TCID | SEQ  | 
+------+----------------------+----------------------+-----+------------+------+------+
|  1   | 192.168.5.10 (50719) | 192.168.5.9 (42042)  |  C  |  FT Start  |  12  | 1036 |
|  2   | 192.168.5.10 (50719) | 192.168.5.9 (42042)  |  D  |            |  12  |  0   |
|  3   | 192.168.5.9 (42042)  | 192.168.5.10 (50719) |  C  | ACK packet |  12  |  0   |
|  5   | 192.168.5.10 (50719) | 192.168.5.9 (42042)  |  D  |            |  12  |  1   |
|  6   | 192.168.5.9 (42042)  | 192.168.5.10 (50719) |  C  | ACK packet |  12  |  1   |
\end{lstlisting}
\end{minipage}
% \end{adjustbox}

\vspace{1ex}
The sending PC starts a file transmission with a file transfer start packet and
the first data packet. The next packet (3) is comming from the FPGA to the PC
and acknowledges the first data packet (sequence 0). To verify that all packets
are acknowledged, the sending program reports an acknowledge status after the
files has been sent. 

\begin{minipage}{1\linewidth}
    \begin{lstlisting}[
        style=TextStyle, 
        % caption=ack buffer allocation, 
        % label=lst:ackbufalloc
        ]
$ ./sender 192.168.5.9 42042 payload/cat.jpg
UFT Sender demo
destination 192.168.5.9:42042
HURRAY! All 1036 packets have been acknowledged.
time elapsed: 1.18s Speed: 0.859 MB/s Size: 1.012 MB\end{lstlisting}
\end{minipage}

\vspace{1ex}
\textbf{User register:} To test wether the user registers can be written and are
output correctly by the communication core, a \gls{ila} 
was used. An ILA can be configured on the FPGA to record internal signals. The
results are transferred to the PC over USB and displayed in \gls{vivadohlx}. All
user registers (0 through 7) are written with differen values and the reult
observed using the ILA matched the sent data.

\vspace{1ex}
\textbf{Stream interface:} The last modification made to the communication core
was the AXI4-Stream interface. This was similarly tested as the user registers.
Using an ILA the output of the stream was observed and the correct order of data
verified. 

\subsubsection*{Controller} 
The most important thing to verify in the controller core is the correct order
of output pixel. This is done by generating input data representing an image,
feeding this data through the controller core and observing the output pixels.
This validation is split into the two solutions implemented using HLS (solution
A) and VHDL (solution B).

\vspace{1ex}
\textbf{Solution A) HLS:} In Vivado HLS the validation was done in C and
Co-simulation. A testbench C/C++ file generates pseudo random image data. This
data is then processed using the controller core and in the testbench itself.
The two results are then compared. Using different input data and image sizes,
all output results matched. A test on the FPGA was omitted. This validation will
later take plave in the overall validation in chapter \ref{ch:verification:overallvalidation}.

\vspace{1ex}
\textbf{Solution B) VHDL:} The same approach was used in the solution written in
VHDL except that the testbench is not written in C/C++ but in VHDL. The input
stream sends an incrementing pixel value with an image width and height of
eight. The window length was reduced to three for simplicity. Figure 
\ref{fig:vhdlcontrollerstimuli} shows the input data values.

 \begin{figure}[h!]
    \centering
    \begin{adjustbox}{max width=\linewidth}
        % \tikzsetnextfilename{system-overview}
\begin{tikzpicture}[
    rounded corners=0mm,
    triangle/.style = {fill=blue!20, regular polygon, regular polygon sides=3 },
    node rotated/.style = {rotate=180},
    border rotated/.style = {shape border rotate=180}
]
    %coordinates
    \coordinate (orig)      at (0,0);

    \begin{pgfonlayer}{main}
        
        % Text
        % \node[] (write) at (-2,5) {Write};
        % \node[] (read) at (0,6.2) {Read};

        % Braces
        \draw [line width=0.5mm,decorate,decoration={brace,amplitude=10pt},xshift=-4pt,yshift=0pt] (4.5,4) -- (4.5,0) node [black,midway,xshift=0.5cm,anchor=west] {Height};
        \draw [line width=0.5mm,decorate,decoration={brace,amplitude=10pt},xshift=-0pt,yshift=4pt] (4,-0.5) -- (0,-0.5) node [black,midway,yshift=-0.5cm,anchor=north] {Width};
        
        % Center pixel
        \draw[black,line width=0.5mm] (0.5,3.0) rectangle (1,3.5);
        % Window size
        \draw[black,line width=0.5mm] (0,2.5) rectangle (1.5,4.0);

        % Axis
        \foreach \x in {0,1,2,3,4,5,6,7}
            \node[anchor=south] at ($(0.25,0) + (0,3.5)+0.5*(\x,0)$)  {0\padzeroes[1]\Hexadecimalnum{\x}};
        \foreach \x in {8,9,10,11,12,13,14,15}
            \node[anchor=south] at ($(0.25,0) + (-4,3.0)+0.5*(\x,0)$)  {0\padzeroes[1]\Hexadecimalnum{\x}};
        \foreach \x in {16,17,18,19,20,21,22,23}
            \node[anchor=south] at ($(0.25,0) + (-8,2.5)+0.5*(\x,0)$)  {\xintDecToHex{\x}};
        \foreach \x in {24,25,26,27,28,29,30,31}
            \node[anchor=south] at ($(0.25,0) + (-12,2)+0.5*(\x,0)$)  {\xintDecToHex{\x}};
        \foreach \x in {32,33,34,35,36,37,38,39}
            \node[anchor=south] at ($(0.25,0) + (-16,1.5)+0.5*(\x,0)$)  {\xintDecToHex{\x}};
        \foreach \x in {40,41,42,43,44,45,46,47}
            \node[anchor=south] at ($(0.25,0) + (-20,1.0)+0.5*(\x,0)$)  {\xintDecToHex{\x}};
        \foreach \x in {48,49,50,51,52,53,54,55}
            \node[anchor=south] at ($(0.25,0) + (-24,0.5)+0.5*(\x,0)$)  {\xintDecToHex{\x}};
        \foreach \x in {56,57,58,59,60,61,62,63}
            \node[anchor=south] at ($(0.25,0) + (-28,0.0)+0.5*(\x,0)$)  {\xintDecToHex{\x}};

    \end{pgfonlayer}

    % Foreground
    \begin{pgfonlayer}{foreground}
        
    \end{pgfonlayer} 

    % Background
    \begin{pgfonlayer}{background}
        % Grid
        \draw[step=0.5cm,gray] (0,0) grid (4,4);
    \end{pgfonlayer} 

\end{tikzpicture}
    \end{adjustbox}
    \caption{VHDL controller validation stimuli}
    \label{fig:vhdlcontrollerstimuli}
\end{figure}

Observed was the output stream that was stored in a text file by the VHDL
testbench. The resulting pixel order was then validated using random samples.
Listing \ref{lst:controlleroutstreamvhdl} shows the output pixel values of the
first three lines.

\begin{minipage}{\linewidth}
    \begin{lstlisting}[
        style=TextStyle, 
        caption=Output stream hexadecimal coded, 
        label=lst:controlleroutstreamvhdl
        ]
00 08 10 01 09 11 02 0A 12 03 0B 13 04 0C 14 05 0D 15 06 0E 16 07 0F 17 
08 10 18 09 11 19 0A 12 1A 0B 13 1B 0C 14 1C 0D 15 1D 0E 16 1E 0F 17 1F
10 18 20 11 19 21 12 1A 22 13 1B 23 14 1C 24 15 1D 25 16 1E 26 17 1F 27\end{lstlisting}
\end{minipage}

% ==============================================================================
%
%                             Overall Validation
%
% ==============================================================================
\subsection{Overall Validation}\label{ch:verification:overallvalidation}
To validate the overall system, the image processing and dataflow parts are
combined in a Vivado HLx project. Two projects are distinguished:
\begin{itemize}
    \item \texttt{diip}\footnote{\Gls{diip}} Project using solution A) with HLS
    implementation
    \item \texttt{diip\_faster} Project using solution B) with VHDL
    implementation
\end{itemize}

On the computer side, the program \texttt{diip\_cc} written in C++ reads the
pixels from an input image, splits the data into according UFT packets and sends
them to the FPGA. The processed pixels are received and stored in the output
image. Similar to the proceeding in \ref{ch:verification:imageprocessing} two
test images were used and compared to the results of a floating point
calculation on the computer. The resulting \gls{rmse} is listed in table \ref{tab:overallvalidationresults}. 

\todo[inline]{Fill table}
\begin{table}[tb!]
    \centering
    \begin{tabular}{l c c}
        \toprule
        Solution & RMSE room & RMSE mountain \\
        \midrule
        diip (HLS) & 0.001 & 0.001 \\
        diip\_faster (VHDL) & 0.001 & 0.001 \\
        \bottomrule
    \end{tabular}
    \caption{Overall validation results}
    \label{tab:overallvalidationresults}
\end{table}


% ==============================================================================
%
%                             Benchmark
%
% ==============================================================================
\section{Benchmark}\label{ch:benchmark}

